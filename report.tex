\documentclass[a4paper,12pt]{report}
\usepackage[utf8]{vietnam}
\usepackage{graphicx}
\usepackage{fancybox}
\usepackage{longtable}
\usepackage{listings}
\usepackage{relsize}
\usepackage[left=3cm, right=2.00cm, top=2.00cm, bottom=2.00cm]{geometry}
\lstset{
   %keywords={break,case,catch,continue,else,elseif,end,for,function,
   %   global,if,otherwise,persistent,return,switch,try,while},
   basicstyle=\ttfamily \fontsize{12}{15}\selectfont,   
	% numbers=left,
   frame=lrtb,
tabsize=2
}
\usepackage{hyperref}  
\usepackage{float}
\hypersetup{
    colorlinks,
    citecolor=black,
    filecolor=black,
    linkcolor=black,
    urlcolor=black
}
\usepackage[nottoc]{tocbibind}
\usepackage[english]{babel}
\addto\captionsenglish{%
 \renewcommand\chaptername{Phần}
 \renewcommand{\contentsname}{Mục lục} 
 \renewcommand{\listtablename}{Danh sách bảng}
 \renewcommand{\listfigurename}{Danh sách hình vẽ}
 \renewcommand{\tablename}{Bảng}
 \renewcommand{\figurename}{Hình}
}
\begin{document}
\thispagestyle{empty}
\thisfancypage{
\setlength{\fboxrule}{1pt}
\doublebox}{}
\begin{center}
{\fontsize{16}{19}\fontfamily{cmr}\selectfont TRƯỜNG ĐẠI HỌC BÁCH KHOA HÀ NỘI\\
VIỆN CÔNG NGHỆ THÔNG TIN VÀ TRUYỀN THÔNG}\\
\textbf{------------*******---------------}\\[1cm]
\includegraphics[scale=0.13]{hust.jpg}\\[1.3cm]

{\fontsize{32}{43}\fontfamily{cmr}\selectfont BÁO CÁO}\\[0.1cm]
{\fontsize{38}{45}\fontfamily{cmr}\fontseries{b}\selectfont MÔN HỌC}\\[0.2cm]
{\fontsize{19}{20}\fontfamily{phv}\selectfont Học máy }\\[0.2cm]
{\fontsize{13}{20}\fontfamily{cmr}\selectfont Đề tài: So sánh thử nghiệm các phương pháp học máy\\ cho bài toán phân loại ảnh.
}\\[2.5cm]
\end{center}
\hspace{1cm}\fontsize{14}{16}\fontfamily{cmr}\selectfont \textbf{Nhóm sinh viên thực hiện:}

\begin{longtable}{l c c}

Họ và tên & MSSV  & Lớp\\
Nguyễn Tuấn Đạt & 20130856 & CNTT2.02-K58 \\
Vũ Minh Đức & 20130856 & CNTT2.02-K58 \\
Nguyễn Ngọc Huyền & 20130856 & CNTT2.02-K58 \\
Đặng Quang Trung & 20130856 & CNTT2.02-K58 \\
Phan Anh Tú & 20130856 & CNTT2.02-K58 \\

\end{longtable}

\hspace{1cm}\fontsize{14}{16}\fontfamily{cmr}\selectfont \textbf{Giáo viên hướng dẫn: }TS.Thân Quang Khoát \\[1.5cm]
\begin{center}
\fontsize{16}{19}\fontfamily{cmr}\selectfont Hà Nội 12--2016

\end{center}
\newpage
\pdfbookmark{\contentsname}{toc}
\tableofcontents
\chapter*{Lời cảm ơn}
\phantomsection
\addcontentsline{toc}{chapter}{Lời cảm ơn}

\listoffigures
\chapter{Mở đầu}
\chapter{Giới thiệu bài toán}
\section{Giới thiệu bài toán}
\section{Bộ dữ liệu sử dụng}
\chapter{Các phương pháp sử dụng và kết quả thực nghiệm}
\section{KNN}
\subsection{Cơ sở lý thuyết}
\begin{itemize}
\item Giai đoạn học
\begin{itemize}
\item KNN - K nearest neighbors : là một phương pháp học máy dựa trên việc lưu lại các các ví dụ học trong tập dữ liệu training.
\end{itemize}
\item Giai đoạn phân lớp
\begin{itemize}
\item Dùng một hàm để tính độ tương đồng giữa các ví dụ traning đã lưu và dữ liệu từ bộ test.
\item Lưu lại k ví dụ có độ tương đồng với dữ liệu test nhất, từ đó dự đoán nhãn cho ví dụ test đầu vào theo lớp chiếm số đông trong số các lớp của k láng giềng. 
\end{itemize}
\item Vấn đề cần giải quyết với giải thuật KNN
\begin{itemize}
\item Có nhiều hàm tính độ tương đồng, cần lựa chọn và thử nghiệm để chọn ra hàm tương đồng phù hợp với bộ dữ liệu.
\item Lấy bao nhiêu hàng xóm cho đủ. 
\end{itemize}
\end{itemize}
\subsection{Cài đặt}
\begin{itemize}
\item Lựa chọn mô hình
\begin{itemize}
\item Hàm tính độ tương đồng  :D=\begin{align*} \textbar x
 \end{align*}
\begin{itemize}
\end{itemize}
(ghi chú: cần nêu rõ cấu trúc mã nguồn, chương trình, vai trò của các lớp và các phương thức chính)

\subsection{Kết quả}

\section{Mạng neural}
\section{CNN}
\section{SVM}
\chapter{Kết luận}
\section{So sánh các phương pháp}
\section{Khó khăn gặp phải}
\section{Kinh nghiệm rút ra được}
\chapter{Tài liệu tham khảo}
\begin{verbatim}
[+] https://

\end{verbatim}


\end{document}